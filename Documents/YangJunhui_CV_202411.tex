%------------------------------------
% Dario Taraborelli
% Typesetting your academic CV in LaTeX
%
% URL: http://nitens.org/taraborelli/cvtex
% DISCLAIMER: This template is provided for free and without any guarantee 
% that it will correctly compile on your system if you have a non-standard  
% configuration.
% Some rights reserved: http://creativecommons.org/licenses/by-sa/3.0/
%------------------------------------

%!TEX TS-program = xelatex
%!TEX encoding = UTF-8 Unicode

\documentclass[11pt, a4paper]{article}
\usepackage{fontspec} 

% DOCUMENT LAYOUT
\usepackage{geometry} 
\geometry{a4paper, textwidth=5.35in, textheight=10.1in, marginparsep=7pt, marginparwidth=.6in, right=0.8in, left=1.25in}
\setlength\parindent{0in}

% FONTS
\usepackage[usenames,dvipsnames]{xcolor}
\usepackage{xunicode}
\usepackage{xltxtra}
\defaultfontfeatures{Mapping=tex-text}
%\setromanfont [Ligatures={Common}, Numbers={OldStyle}, Variant=01]{Linux Libertine O}
%\setmonofont[Scale=0.8]{Monaco}
%%% modified by Karol Kozioł for ShareLaTeX use
\setmainfont[
  Ligatures={Common}, Numbers={OldStyle}, Variant=01,
  BoldFont=LinLibertine_RB.otf,
  ItalicFont=LinLibertine_RI.otf,
  BoldItalicFont=LinLibertine_RBI.otf
]{LinLibertine_R.otf}
\setmonofont[Scale=0.8]{DejaVuSansMono.ttf}

% ---- CUSTOM COMMANDS
\chardef\&="E050
\newcommand{\html}[1]{\href{#1}{\scriptsize\textsc{[html]}}}
\newcommand{\pdf}[1]{\href{#1}{\scriptsize\textsc{[pdf]}}}
\newcommand{\doi}[1]{\href{#1}{\scriptsize\textsc{[doi]}}}
% ---- MARGIN YEARS
\usepackage{marginnote}
\newcommand{\amper{}}{\chardef\amper="E0BD }
\newcommand{\years}[1]{\marginnote{\scriptsize #1}}
\renewcommand*{\raggedleftmarginnote}{}
\setlength{\marginparsep}{7pt}
\reversemarginpar

% HEADINGS
\usepackage{sectsty} 
\usepackage[normalem]{ulem} 
\sectionfont{\mdseries\upshape\Large}
\subsectionfont{\mdseries\scshape\normalsize} 
\subsubsectionfont{\mdseries\upshape\large} 

% PDF SETUP
% ---- FILL IN HERE THE DOC TITLE AND AUTHOR
\usepackage[%dvipdfm, 
bookmarks, colorlinks, breaklinks, 
% ---- FILL IN HERE THE TITLE AND AUTHOR
	pdftitle={YangJunhui\_CV\_202411.pdf},
	pdfauthor={Junhui Yang},
	%pdfproducer={http://nitens.org/taraborelli/cvtex}
]{hyperref}  
\hypersetup{linkcolor=blue,citecolor=blue,filecolor=black,urlcolor=MidnightBlue} 
\newcommand\tab[1][1cm]{\hspace*{#1}}

% DOCUMENT
\begin{document}
\begin{flushright}
\scriptsize{Last updated: \today}
\end{flushright}
{\LARGE Junhui Yang}\\[.3cm]
{\small Rotman School of Management, University of Toronto \\
105 St George St, Toronto, ON M5S 3E6, Canada}\\[.1cm]
{\small Email: \href{mailto: junhui.yang@rotman.utoronto.ca}{junhui.yang@rotman.utoronto.ca}} \\[.1cm]
{\small Personal Website: \href{https://junhui-yang.com}{https://junhui-yang.com}

\section*{Research Interests}
Urban Economics • Economic Geography • Empirical Industrial Organization

%\hrule
\section*{Education}
\noindent
\years{2019-2025}{\small Ph.D. Management, Concentration in Economic Analysis and Policy, University of Toronto}\\[.1cm]
	{\small \tab Committee: Nathaniel Baum-Snow (co-supervisor), Stephan Heblich (co-supervisor), William C. Strange}\\[.1cm]
\years{Spring 2024}{\small Visiting Research Student, London School of Economics (LSE)}\\[.1cm]
\years{2017-2019}{\small M.A. Economics and Finance, Centro de Estudios Monetarios y Financieros (CEMFI), Spain}\\[.1cm]
\years{2013-2017}{\small B.A. Economics, Xiamen University (XMU), China}\\[.1cm]
\years{Spring 2016}{\small Exchange Student, Universidad Carlos III de Madrid (UC3M), Spain}

\section*{Research}
{\small ``Land Use Regulation as a Barrier to Entry: Evidence from Minimum Parking Requirements in Retail and Local Services'' (Job Market Paper) \\[.15cm]
{\small ``Bus Network Redesign, Commuting and Welfare: Evidence from Houston'' (working paper, revise and resubmit at \textit{Journal of Urban Economics})} \\[.1cm]
{\small ``Ice Roads'' with Victor Aguirregabiria and Stephan Heblich (work in progress)}

%%\hrule
%\section*{Work in Progress}

\section*{Policy Article}
{\small ``Competition in Canada from 2000 to 2020: An Economy at a Crossroads'' with Ramin Fourouzandeh, Matthew Osborne and Farhang Shamsoddin} 

\section*{Teaching Assistantship}
{\small RSM 483: Real Estate Markets (undergraduate, $\times$ 3)}\\[.1cm]
{\small RSM 1210: Managerial Economics (MBA)}\\[.1cm]
{\small RSM 2122: Clean Energy: Policy Context and Business Opportunities (MBA)}\\[.1cm]
{\small RSM 2128: Real Estate Economics (MBA)}\\[.1cm]
{\small MGE B11: Quantitative Methods in Economics I (undergraduate, $\times$ 2)}\\[.1cm]
{\small MGE C06: Topics in Macroeconomic Theory (undergraduate)}\\[.1cm]
{\small MGE D02: Advanced Microeconomic Theory (undergraduate)}\\[.1cm]
{\small MGE D11: Theory and Practice of Regression Analysis (undergraduate)}\\[.1cm]
{\small GLA 2071: Topics in Markets III: Environmental Economics (graduate, $\times$ 2)}\\[.1cm]
{\small GLA 2081: Topics in Innovation II: Technology Policy (graduate, $\times$ 2)}

\section*{Research Assistantship}
\years{2021-2023}{\small Stephan Heblich, University of Toronto}\\[.1cm]
\years{2021}{\small Yue Yu, University of Toronto}\\[.1cm]
\years{2018}{\small Diego Puga, CEMFI}

%%\hrule
\section*{Honors \& Awards}
\noindent
\years{2024}{University of Toronto Doctoral Completion Award}\\[.1cm]
\years{2024}{University of Toronto International Experience Award +}\\[.1cm]
\years{2023}{PhD Student Fellow, Center for Real Estate and Urban Economics, Rotman School of Management}\\[.1cm]
\years{2019-2025}{University of Toronto Doctoral Fellowship}\\[.1cm]
\years{2017-2019}{CEMFI Master's Program Full Scholarship}

\section*{Conferences \& Talks}
\noindent
\years{2024}{\small LSE Economic Geography Seminar (London)}\\[.1cm]
\years{2022}{\small Urban Economics Association Summer School (Barcelona)}\\[.1cm]
\years{2022}{\small European Meeting of Urban Economics Association (London)}
%%\hrule
\section*{Referee Service}
\noindent
{\small Journal of Urban Economics ($\times$ 2)}

%\hrule
\section*{Data Clearance}
Research Data Center, Statistics Canada

\section*{Skills}
Languages: Mandarin Chinese (native), English (fluent)\\[.1cm]
Programming: Python, R, SQL, PySpark (in DataBricks), AWS SageMaker, Azure, MATLAB, ArcGIS, Julia, Stata

\section*{References}
\noindent\begin{tabular}{@{}lll}
Nathaniel Baum-Snow                 &  Stephan Heblich & William C. Strange           \\
Rotman School of Management   &  Department of Economics & Rotman School of Management  \\
University of Toronto     &  University of Toronto  & University of Toronto  \\
105 St\ George St        & 150 St George St & 105 St George St       \\
Toronto, Ontario          & Toronto, Ontario  & Toronto, Ontario        \\
 M5S 3E6, Canada           & M5S 3G7, Canada  & M5S 3E6, Canada          \\
nate.baum.snow@rotman.utoronto.ca      & stephan.heblich@utoronto.ca & wstrange@rotman.utoronto.ca\\
+1-416-978-4273           & +1-416-978-4622   & +1-416-978-1949        \\\end{tabular}

\newpage
\begin{center}
\LARGE
\textbf{Abstracts}
\normalsize
\end{center}
\line(1,0){505}

\begin{center}
\LARGE
\textbf{Land Use Regulation as a Barrier to Entry: Evidence from Minimum Parking Requirements in Retail and Local Services}\\
\large
(Job Market Paper)
\normalsize
\end{center}

I study the cost impacts of minimum parking requirement (MPRs) for retail and local services firms. As a common North American land use regulation, MPRs require firms to provide parking proportional to size upon opening and operating. I derive exogenous sources of variation from a 2012 reform in Seattle that reduced MPRs in only parts of the city and created arbitrary boundaries. I find extensive margin effects of MPR reductions, which positively impacted firm entry, survival, and related local outcomes, implying MPRs increase firm entry and fixed costs. I build a dynamic game model of entry and exit to quantify, and I find MPRs raise entry costs by 24 percent. A census tract would have at least one percent more firms if its MPRs were removed. \\

\bigskip
\begin{center}
\LARGE
\textbf{Bus Network Redesign, Commuting and Welfare: Evidence from Houston}\\
\large
(working paper, revise and resubmit at \textit{Journal of Urban Economics})
\normalsize
\end{center}
In August 2015, Houston rolled out a complete redesign of its bus route network. This came in response to falling ridership, a new light rail system, and new employment centers. This paper evaluates the consequences for commuting patterns and welfare. Using old bus routes as the instrumental variable, at the census tract pair level, I find a one-minute travel time reduction due to the redesign led to a 0.94\% increase in the number of commuters from 2014 to 2018. I also find large extensive margin effects. Evaluated through a quantitative spatial equilibrium model with mode choice, the redesign was welfare-improving. My counterfactual exercises bring new insights on the importance of connecting buses to intermodal transfer points, population and employment centers. My contributions include that I provide a first study of the bus to the transit policy evaluation literature, and that I study a cheap re-optimization, instead of a costly expansion, of transit.

\bigskip
\begin{center}
\LARGE
\textbf{Ice Roads}\\
\large
with Victor Aguirregabiria and Stephan Heblich\\
(work in progress)
\normalsize
\end{center}
This paper examines the impact of global warming on the presence of ice roads in Northern Canada, which are crucial for transporting essential goods to remote Northern communities, particularly First Nations. In recent decades, the operational duration of ice roads has diminished, resulting in a higher reliance on costly supply flights. To assess the consequences on food security, we exploit data from Nutrition North Canada (NNC), which offers subsidies for shipping costs by air, ice road, or sea to eligible communities. As expected, the reduced availability of ice roads has led to higher prices and lower consumption. Our study employs these insights to simulate potential future scenarios driven by ongoing climate change and evaluates the costs and benefits of policy measures aimed at mitigating the impact of dwindling winter roads. \\

\end{document}